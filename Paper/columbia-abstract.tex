\documentclass[notitlepage]{article}
\author{Zachary Sussman}
\title{Automatic Particle Detection in Cloud Chambers}
\date{}

\usepackage{amsmath}
\usepackage[version=3]{mhchem}
\usepackage{chemfig}
\usepackage{enumitem}
\usepackage{siunitx}
\sisetup{
	free-standing-units=true,
	space-before-unit=false,
	use-xspace=true
}

\usepackage[procnames]{listings}
\usepackage{color}

\usepackage{hanging}

\usepackage{graphicx}
\graphicspath{ {images/} }

\usepackage{caption}
\usepackage{subcaption}

\usepackage[title,titletoc,toc]{appendix}

\usepackage{setspace}

%\usepackage[margin=1in]{geometry}

\begin{document}

\definecolor{keywords}{RGB}{255,0,90}
\definecolor{comments}{RGB}{0,0,113}
\definecolor{red}{RGB}{160,0,0}
\definecolor{green}{RGB}{0,150,0}
 
\lstset{language=Python, 
        basicstyle=\ttfamily\small, 
        keywordstyle=\color{keywords},
        commentstyle=\color{comments},
        stringstyle=\color{red},
        showstringspaces=false,
        identifierstyle=\color{green},
        procnamekeys={def,class}}



\maketitle

\pagenumbering{gobble}
\begin{abstract}
\noindent
\textbf{Introduction:} A cloud chamber is a user-friendly particle detector utilized for smaller-scale research and educational purposes.  One major limitation of this type of device is the lack of accessible automated programs to detect and analyze the particles. I attempted to rectify this situation by creating my own algorithms to detect and evaluate subatomic particle tracks in videos of functioning cloud chambers.
\\ \textbf{Methods:} I constructed a 10 gallon diffusion cloud chamber and collected 5 hours of video.  Using the OpenCV video-processing library for Python, I created a program that analyzed the video and generated a database of track events and properties. The program runs frame by frame, finding groups of foreground pixels, which correspond to tracks.  By matching tracks in sequential frames, a database of events is built. Each type of subatomic particle leaves a different path in the chamber, so by evaluating track characteristics including aspect ratio, length, and intensity, I can identify particle type and approximate its energy.
\\ \textbf{Results:} With just 200 lines of Python, the program processed tracks at a 90\% accuracy rate, and identified beta particle energies within 0.5-1 \kilo{}\electronvolt and alpha particle energies within 10-20 \kilo{}\electronvolt. 
\\ \textbf{Conclusion:} With fairly straightforward algorithms, my program can identify particle tracks in cloud chambers with good accuracy. The automation of this cost-effective particle detector removes a barrier to more widespread use.


\end{abstract}

\clearpage

This project was completed for my Research class over 3 months from September to December.  Because of scheduling conflicts with band, I could not attend the class, so I got an exemption to complete this research project and paper as an independent project.  I tried to run the cloud chamber at school, but the logistics did not work out, so the entire project was completed at my house.  The particle detection program is my original work.

\end{document}